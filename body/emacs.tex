
\chapter{Emacs}
\label{chap:emacs}

\begin{flushleft}
\rule{\textwidth}{0.1pt}
\end{flushleft}

\texttt{vi}显然是类Unix中存在最为广泛的编辑器,而Emacs仅次之。有一种说
法是\texttt{vi}是编辑的神器,而\texttt{Emacs}则是神的编辑器。为什么呢?
\texttt{vi}中是使用不同的``模式''来方便我区分文本的处理和文本的输入,
而\texttt{Emacs}则通过使用Ctrl和Alt键的一些键绑定来达到此目的。这和现
在我们在其它一些字处理程序中的快捷键的概念相似。(尽管如此,我们还是要
强调,\texttt{Emacs}与这些字处理程序的快捷键不是对应的,如传统上分别用
Ctrl+C/X/V表示复制、剪切和粘贴,而Emacs中使用的键则与此不同。)

另外,\texttt{vi}就只是一个编辑器,除此之外再无其它。而\texttt{Emacs}
则是一个无所不包的程序。由于\texttt{Emacs}(大部分)是用Lisp语言写的,
Lisp语言是一个非常强大的语言,它还有个诡异的特性,就是所有用Lisp语言写
的程序本身就是一个Lisp语言的编译器\footnote{Every program written in
  it is automatically a Lisp compiler of its own.不太了解Lisp,不知道
  这句话的含义。}。这意味着我们可以对Emacs进行扩展,实际上可以``在
Emacs''中写全新的程序。

因此,Emacs不再仅仅是一个编辑器。甚至有人说Emacs是个操作系统。网上有许
多Emacs的插件(许多在程序源码中就自带了),为我们提供各种各样的功能。
这些插件的很大一部分与编辑文本有关,毕竟这才是Emacs的本职工作。但不仅
如此,例如,有一些插件是让Emacs具有表格编辑功能。还有数据库、游戏、邮
件及新闻的客户端(最牛的当属Gnus)等等。

当前有两个主流的Emacs版本:GNU Emacs(Slackware自带的即为该版本)和
XEmacs。XEmacs最初的动机是清理Emacs代码的一个项目。当前,两个版本都发
展良好,事实上还有试图将两者结合起来的开发小组。本章中介绍的内容对于两
个版本都没有区别。它们的区别于普通用户无关。

\section{启动Emacs}
\label{sec:emacs:start}
Emacs可以在命令行中启动,键入\texttt{emacs}即可启动。如果你在X环境下,
Emacs(正常情况下)会开启自己的X窗口。启动后,Emacs首先会显示一个欢迎
信息,几秒后会进入*scratch*缓冲区。(见第\ref{sec:emacs:buffers}节)

当然,也可以在启动Emacs时就指定打开一个已经存在的文件:
\begin{Verbatim}[frame=single,commandchars=\\\{\}]
\% \textbf{emacs /etc/resolv.conf}
\end{Verbatim}
这会使Emacs跳过欢迎信息,而载入指定的文件。

\subsection{命令键}
\label{sec:emacs:start:commandKeys}
像之前提到的一样,Emacs使用Ctrl及Alt键和其它键的组合键来表示命令。一般
约定分别写为C-字符及M-字符。所以C-x代表Ctrl+x,而M-x则代表Alt+x。(使
用M而不是A的原因是Alt键最早称作Meta键。而Meta键最终在现代键盘上消失了,
Emacs使用Alt键来替代它的功能。)
许多Emacs命令包含多个键入键组合。例如:C-x C-c(先按下Ctrl+x,再按下
Ctrl+c)用来退出Emacs,而C-x C-s用来保存当前文件。请记住,C-x C-b与C-x
b代表不同的键。前一个是按下Ctrl+x后按下Ctrl+b键,而后者是先按下C-x,之
后只按下`b'键。

\section{缓冲区}
\label{sec:emacs:buffers}
在Emacs中,``缓冲区''的概念是很重要的。每个我们打开的文件都会被载入到
它自己的缓冲区中。另外,Emacs还有许多特殊缓冲区,它们不包含文件而是有
其它的作用。而这些特殊缓冲区的名字一般以星号(`*')开头,星号结尾。例如,
Emacs启动时进入的缓冲区为*scratch*。我们可以在其中输入任意的内容,但在
Emacs退出时,这些内容不会被保存。

我们还要了解的一个特殊缓冲区就是minibuffer。该缓冲区只有一行,且总是显
示在屏幕上:它在Emacs窗口中最下面的那行,在当前缓冲的状态栏下面。
minibuffer是Emacs用来为用户显示信息的,并且,当一个命令需要额外输入时,
也是在minibuffer中输入的。例如,我们用C-x C-f命令打开一个新文件,Emacs
会要求我们在minibuffer中输入要打开的文件名。

从一个缓冲区切换到另一个缓冲区使用命令C-x b。它会在minibuffer上提示当
前的所有缓冲区的名字(一般缓冲的名字就是当前编辑的文件名),它会给出一
个默认缓冲区,通常就是最近编辑过的缓冲。选定后按回车即可。

如果我们想切换的缓冲不是默认的,那么我们就要手工输入要切换到的缓冲区的
名字。这里别忘了我们提到过的TAB补全:只要输入开始的几个字符,按下TAB,
Emacs就会尽可能地为我们补全。Emacs的一些其它有意义的地方也能使用补全。

我们可以通过键入C-x C-b来得到当前打开缓冲的一个列表。这个命令通常会将
屏幕分成两半,在上半部分显示我们工作的缓冲区,而下半部分则是名为
*Buffer List*的缓冲区。这个缓冲区中包含了当前打开的缓冲区列表,对应的
大小、模式以及缓冲查看的文件。我们可以按下C-x 1来关闭它。

\begin{quote}
  \textbf{注意:}如果在X环境下,Emacs的菜单中也有当前所有打开缓冲区的
  列表。
\end{quote}

\section{模式}
\label{sec:emacs:modes}
每个Emacs的缓冲区都有对应的模式。这里的模式与\texttt{vi}中的模式有着天
壤之别:Emacs中的模式是告诉我们我们处在什么样的缓冲区中。例如,对于普
通的文本文件,我们有文本模式(text-mode),但还有其它的模式,如C模式
(c-mode)用于编辑C程序,sh模式(sh-mode)用于编辑shell脚本。而latex模
式(latex-mode)则用于编辑\LaTeX{}文件,而mail模式(mail-mode)用于编
辑邮件及新闻信息,等等。每个模式都为我们作了特殊的定制,以用来编辑特殊
的文本。我们还可以为一个模式定义它特有的键及键命令。例如,在文本模式
(text-mode)下,TAB键是用来跳到下个制表符,但在许多编程语言的模式下,
TAB键就被定义为根据当前代码的位置进行缩进。

我们之前提到的模式称为主模式(major mode)。每个缓冲区只在处于一个主模
式下,但可以有一个或多个副模式(minor mode),副模式为我们提供了一些额
外的编辑功能。例如,如果我们按下``Esc''键,我们就会调用覆盖模式
(overwrite-mode),这是你想不到的吧。还有一个叫auto-fill-mode,它
和text-mode及latex-mode结合起来很方便:当我们在一行中输入字符到一定个数,
它会自动跳到下一行。如果没有auto-fill-mode,我们要手工调
用\texttt{M-q}来对一个段落进行重排。

\section{打开文件}
\label{sec:emacs:openFiles}
要在Emacs中打开一个文件,键入:
\begin{Verbatim}[frame=single,commandchars=\\\{\}]
\textbf{C-x C-f}
\end{Verbatim}
Emacs会提示我们输入要打开的文件的文件名,还会填入一些默认的路径(通常
是\path{\~/})。输入文件名后(可以使用补全)输入回车,Emacs会在一个新的
缓冲区中打开文件,并在屏幕上显示该缓冲区。
\begin{quote}
  \textbf{注意:}Emacs会自动创建一个新的缓冲区,而不会将文件载入到当前
  缓冲中。
\end{quote}

要在Emacs中创建一个新的文件,并不能像在\texttt{vi}中可以直接输入内容,
最后再保存为文件,相反的,我们必须先为该文件创建一个缓冲区,并为其命名。
我们可以按下C-x C-f,并输入我们想创建的文件的名字,就像在打开一个已经
存在的文件一样。Emacs会发现我们的文件不存在,并为它创建一个新的缓冲,
在minibuffer中显示``(New file)''信息。

如果我们在键入C-x C-f之后输入的是一个目录名而不是文件名,则Emacs会创建
一个新的缓冲区,其中包括了该目录下所有文件的一张列表。我们可以将光标移
动到想打开的文件上,输入回车,则Emacs就会打开它。(当然,这里还可以执
行很多其它的操作,如删除文件、重命名、移动文件等等。这是Emacs的
dired-mode,简单地说就是一个文件管理器。)

如果你按了C-x C-f,突然又不想打开文件了,可以按下C-g来取消这个动作。在
Emacs中,几乎可以在任何地方使用C-g来取消当前的动作。

\section{基本编辑命令}
\label{sec:emacs:basicEditing}
我们打开文件后,想做的当然是移动光标。方向键、鼠标及PageUp、PageDown键
和正常情况的行为一致。Home和End键分别跳转到当前行的行首和行尾。(老版
本中的作用是跳到缓冲区开头和缓冲区末尾。),但我们还有一些Ctrl和
Meta(Alt)的组合键可以完成相同的功能。因为我们在输入时是不会想把手移到键
盘的另一边的。也正是因为我们不必移动我们的手,使用这些键的效率往往更高。
最重要的一些命令已经在表\ref{tab:basic-emacs-editing}列出了。

\begin{table}[htpb]
  \centering
  \begin{tabular}{c|l}
    \hline\hline
    命令 & 作用 \\ \hline
    C-b & 移动到后一个字符的位置(后,指朝文件开始的方向)\\
    C-f & 移动到前一个字符的位置(前,指朝文件末尾的方向) \\
    C-n & 向下移动一行 \\
    C-p & 向上移动一行 \\
    C-a & 移动到当前行行首\\
    C-e & 移动到当前行行末 \\
    M-b & 向后移动一个词\\
    M-f & 向前移动一个词\\
    M-\} & 向前移动一个段落\\
    M-\{ & 向后移动一个段落 \\
    M-a & 移动到当前句子的开头\\
    M-e & 移动到当前句子的末尾\\
    C-d & 删除当前光标所在位置的字符\\
    M-d & 删除到当前单词的末尾\\
    C-v & 向下滚动一屏(即PageDown)\\
    M-v & 向上滚动一屏(即PageUp)\\
    M-$<$ & 移动到当前缓冲区的起始位置 \\
    M-$>$ & 移动到当前缓冲区的结束位置\\
    C-\_ & 撤消最后一次发动(可重复执行)\\
    C-k & 删除直到当前行末的所有字符\\
    C-s & 向前搜索\\
    C-r & 向后搜索\\
    \hline\hline
  \end{tabular}
  \caption{Emacs基本编辑命令}
  \label{tab:basic-emacs-editing}
\end{table}

注意,一般而言Meta的组合键的功能与Ctrl的组合键类似,只是使用的单位更大
一些:如C-f的作用是向前移动一个字符,而M-f则是向前移动一个单词,以此类
推。

另外,如M-$<$和M-$>$这类的组合键分别要求我们同时按下Shif+Alt+逗号及
Shift+Alt+点号。这是由于$<$及$>$分别需要由Shift+逗号键及Shift+点号键才能输
出。(当然,除非你的键盘不是标准美式键盘。)

还要注意的是C-k会删除(Emacs中称为kill)光标后到行末的所有文本。但不删
除换行符本身。如果当前光标后没有其它字符了,它才会删除换行符。也就是如
果想删除一整个行,则要按两下C-k:第一下用于删除多余的文本,第二下删除
换行符本身。

\section{保存文件}
\label{sec:emacs:savingFiles}
要保存一个文件,按键如下:
\begin{Verbatim}[frame=single,commandchars=\\\{\}]
\textbf{C-x C-s}
\end{Verbatim}
Emacs不会询问要保存的文件名,它会保存当时打开的那个文件中。如果你想
``另存为'',可以按如下键:
\begin{Verbatim}[frame=single,commandchars=\\\{\}]
\textbf{C-x C-w}
\end{Verbatim}

在本会话第一次保存文件时,Emacs会将旧的文件保存为一个备份,备份文件名
与原文件相同,只是在文件名之后加上\texttt{\~}符:例如,如果编辑的是
``\path{cars.txt}''文件,则备份文件名为``\path{cars.txt~}''

备份文件是我们打开文件的一个复本。而在我们工作时,Emacs还会为我们创建
一个自动保存的复本,该复本文件名形式如下:\texttt{\#cars.txt\#}。在我们
执行C-x C-s后,Emacs会删除该复本。

当我们编辑结束后,可以使用下面的命令来关闭(kill)一个缓冲区:
\begin{Verbatim}[frame=single,commandchars=\\\{\}]
\textbf{C-x k}
\end{Verbatim}

Emacs会询问我们要关闭哪个缓冲区,默认为当前缓冲,输入回车确认。如果我
们还没有保存,则Emacs会提示我们是否真的希望关闭该缓冲区。

\section{启动Emacs}
\label{sec:emacs:quit}
我们结束编辑任务后,可以按下面的键
\begin{Verbatim}[frame=single,commandchars=\\\{\}]
\textbf{C-x C-c}
\end{Verbatim}
来退出Emacs。如果我们有未保存的文件,Emacs会提示我们,并逐个询问我们是
否保存。如果有任意一个回答no,则Emacs会最后询问一次是否退出,之后退出。




%%% Local Variables: 
%%% mode: latex
%%% TeX-master: "../SlackGuide"
%%% End: 
