
\chapter[GNU通用公共许可证]{GNU通用公共许可证\footnote{ 本节内容的翻译取自
  \url{www.thebigfly.com/gnu/gpl/}。}}
\label{chap:app:GPL}

\section{导言}
\label{sec:app:GPL:preamble}
大多数软件授权声明是设计用以剥夺您共享与修改软件的自由。相反地,GNU通用
公共授权力图保证您分享与修改自由软件的自由-确保软件对所有的使用者都是自
由的。通用公共授权适用于大多数自由软件基金会的软件,以及任何作者指定使
用本授权的其他软件。(有些自由软件基金会的软件,则适用GNU函式库通用公共
授权规定。)您也可以让您的软件适用本授权规定。

当我们在谈论自由软件时,我们所指的是自由,而不是价格。我们的通用公共授
权是设计用以确保使您拥有发布自由软件备份的自由(以及您可以决定此一服务
是否收费),确保您能收到源码或者在您需要时能得到它,确保您能变更软件或
将它的一部分用于新的自由软件;并且确保您知道您可以做上述的这些事情。

为了保障您的权利,我们需要作出限制:禁止任何人否认您上述的权利,或者要
求您放弃这些权利。如果您发布软件的副本,或者对之加以修改,这些限制就转
化成为您的责任。

例如,假如您发布此类程序的副本,无论是免费或收取费用,您必须将您所享有
的一切权利给予收受者。您也必须确保他们也能收到或得到原始程序码。而且您
必须向他们展示这些条款的內容,使他们知到他们所享有的权利。

我们采取两项措施來保护您的权利:(1)以著作权保护软件,以及(2)提供您本授
权,赋与您复制、发布并且/或者修改软件的法律许可。

同时,为了保护作者与我们(按:指自由软件基金会),我们想要确定每个人都
明白,自由软件是沒有担保责任的。如果软件被他人修改并加以传播,我们需要
其收受者知道,他们所得到的并非原始版本,因此由他人所引出的任何问题对原
作者的声誉将不会有任何的影响。

最后,所有自由软件不断地受到软件专利的威胁。我们希望能避免自由软件的再
发布者以个人名义取得专利授权而使程序专有化的风险。为了防止上述的情事发
生,我们在此明确声明:任何专利都必须为了每个人的自由使用而核准,否则就
不应授与专利。

以下是有关复制、发布及修改的明确条款及条件。

\section{复制、发布与修改的条款与条件}
\label{sec:app:GPL:termsAndConditions}

\begin{enumerate}
\setcounter{enumi}{-1}
\item  %\parbox[t]{0.9\textwidth}{
凡著作权人在其程序或其他著作中声明,该程序或著作会在通用公共授权条款下
发布,本授权对其均有适用。以下所称的"程序",是指任何一种适用通用公共授
权的程序或著作;并且一个"基于本程序的著作",则指本程序或任何基于著作权
法所产生的衍生著作,换言之,是指包含本程序全部或一部的著作,不论是完整
的或经过修改的程序,以及(或)翻译成其他语言的程序(以下"修改"一词包括
但不限于翻译行为在內)。被授权人则称为"您"。

本授权不适用于复制、发布与修改以外的行为;这些行为不在本授权范围内。执
行本程序的行为并不受限制,而本程序的输出只有在其內容构成基于本程序所生
的著作(而非只是因为执行本程序所造成)时,始受本授权拘束。至于程序的输
出內容是否构成本程序的衍生著作,则取决于本程序的具体用途。
%}
\item %\parbox[t]{0.9\textwidth}{
    您可以对所收受的本程序源代码,无论以何种媒介,复制与发布其完整的复
制物,然而您必须符合以下要件:以显著及适当的方式在每一份复制物上发布适
当的著作权标示及无担保声明;维持所有有关本授权以及无担保声明的原貌;并
将本授权的副本连同本程序一起交付予其他任何一位本程序的收受者。
    
您可以对让与复制物的实际行为收取一定的费用,您也可以自由决定是否提供担
保以作为对价的交换。
%}
\item 您可以修改本程序的一个或数个复制物或者本程序的任何部份,以此形成
基于本程序所生的著作,并依前述第一条规定,复制与发布此一修改过的程序或
著作,但您必须符合以下要件:

\begin{enumerate}
\item 您必须在所修改的挡案上附加显著的标示,阐明您修改过这些挡案,以及修改日期。
\item 您必须就您所发布或发行的著作,无论是包含本程序全部或一部的著作,
或者是自本程序或其任何部份所衍生的著作,整体授权所有第三人依本授权规定
使用,且不得因此项授权行为而收取任何费用。
\item 若经过修改的程序在执行时通常以交互方式读取命令时,您必须在最常被
使用的方式下,于开始进入这种交互式使用时,列印或展示以下宣告:适当的著
作权标示及无担保声明(或者声明您提供担保)、使用者可以依这些条件再发布
此程序,以及告知使用者如何浏览本授权的副本。(例外:若本程序本身是以交
互的方式执行,然而通常却不会列印该宣告时,则您基于本程序所生的著作便无
需列印该宣告。)
\end{enumerate}
这些要求对修改过的著作是整体适用的。倘著作中可识別的一部份并非衍生自本
程序,并且可以合理地认为是一独立的、个別的著作,则当您将其作为个別著作
加以发布时,本授权及其条款将不适用于该部分。然而当您将上述部分,作为基
于本程序所生著作的一部而发布时,整个著作的发布必须符合本授权条款的规定,
而本授权对于其他被授权人所为的许可及于著作整体。

因此,本条规定的意图不在于主张或剥夺您对于完全由您所完成著作的权利;应
该說,本条规定意在行使对基于程序所生的之衍生著作或集合著作发布行为的控
制权。

此外,非基于本程序所生的其他著作与本程序(或基于本程序所生的著作)在同
一储存或发布的媒介上的单纯聚集行为,并不会使该著作因此受本授权条款约束。

\item 您可以依前述第一、二条规定,复制与发布本程序(或第二条所述基于本
程序所产生的著作)的目的码或可执行形式,但您必须符合以下要件:
\begin{enumerate}
\item 附上完整、相对应的机器可判读源码,而这些源码必须依前述第一、二条
规定在经常用以作为软件交换的媒介物上发布;或
\item 附上至少三年有效的书面报价文件,提供任何第三人在支付不超过实际发
布源码所需成本的费用下,取得相同源码的完整机器可读复制物,并依前述第一、
二条规定在经常用以作为软件交换的媒介物上发布该复制物;或
\item 附上您所收受有关发布相同源码的报价资讯。(本项选择仅在非赢利发布、
且仅在您依前述b项方式自该书面报价文件收受程序目的码或可执行形式时,始有
适用。)
\end{enumerate}
著作的源码,是指对著作进行修改时适用的形式。对于一个可执行的著作而言,
完整的源码是指著作中所包含所有模组的全部源码,加上相关介面的定义挡,还
加上用以控制该著作编译与安裝的描述。然而,特別的例外情况是,所发布的源
码并不需包含任何通常会随著所执行作业系统的主要组成部分(编译器、核心等
等)而发布的软件(无论以源码或二进位格式),除非该部分本身即附加在可执
行程序中。

若可执行码或目的码的发布方式,是以指定的地点提供存取位置供人复制,则提
供可自相同地点复制源码的使用机会,视同对于源码的发布,然而第三人并不因
此而负有将目的码连同源码一起复制的义务。

\item 除本授权所明示的方式外,您不得对本程序加以复制、修改、再授权或发
布。任何试图以其他方式进行复制、修改、再授权或者发布本程序的行为均为无
效,并且将自动终止您基于本授权所得享有的权利。然而,依本授权规定自您手
中收受复制物或权利之人,只要遵守本授权规定,他们所获得的授权并不会因此
终止。

\item 因为您并未在本授权上签名,所以您无须接受本授权。然而,除此之外您
別无其他修改或发布本程序或其衍生著作的授权许可。若您不接受本授权,则这
些行为在法律上都是被禁止的。因此,藉由对本程序(或任何基于本程序所生的
著作)的修改或发布行为,您表示了对于本授权的接受,以及接受所有关于复制、
发布或修改本程序或基于本程序所生著作的条款与条件。

\item 每当您再发布本程序(或任何基于本程序所生的著作)时,收受者即自动
获得原授权人所授予依本授权条款与条件复制、发布或修改本程序的权利。您不
得就本授权所赋予收受者行使的权利附加任何进一步的限制。您对于第三人是否
履行本授权一事,无须负责。

\item 若法院判决、专利侵权主张或者其他任何理由(不限于专利争议)的结果,
使得加诸于您的条件(无论是由法院命令、协议书或其他方式造成)与本授权规
定有所冲突,他们并不免除您对于本授权规定的遵守。若您无法同时符合依本授
权所生义务及其他相关义务而进行发布,那么其结果便是您不得发布该程序。例
如,若专利授权不允许其他人直接或间接取得复制物,通过您以免付权利金的方
式再发布该程序,您唯一能同时滿足该义务及本授权的方式就是徹底避免进行该
程序的发布。

若本条任一部份在特殊情况下被认定无效或无法执行时,本条其余部分仍应适用,
且本条全部于其他情况下仍应适用。

本条的目的并不在诱使您侵害专利或其他財产权的权利主张,或就此类主张的有
效性加以争执;本条的唯一目的,是在保障藉由公共授权惯例所执行自由软件发
布系统的完整性。许多人信赖该系统一贯使用的应用程序,而对经由此系统发布
的大量软件有相当多的贡献;作者/贡献者有权决定他或她是否希望经由其他的
系统发布软件,而被授权人则无该种选择权。

本条的用意在于将本授权其他不确定的部分徹底解释清楚。

\item 若因为专利或享有著作权保护的介面问题,而使得本程序的发布与/或使
用局限于某些国家时,则将本程序置于本授权规范之下的原著作权人得增列明确
的发布地区限制条款,将这些国家排除在外,而使发布的许可只限在未受排除的
国家之內或之中。在该等情况下,该限制条款如同以书面方式订定于本授权內容
中,而成为本授权的条款。

\item 自由软件基金会得随时发表通用公共授权的修正版与/或新版本。新版本
在精神上将近似于目前的版本,然而在细节上或所不同以因应新的问题或状况。

每一个版本都有个別的版本号码。若本程序指定有授权版本号码,表示其适用该
版本或是"任何新版本"时,您可以选择遵循该版本或任何由自由软件基金会日后
所发表新版本的条款与条件。若本程序并未指定授权版本号码时,您可以选择任
一自由软件基金会所发表的版本。

\item 若您想将部分本程序纳入其他自由程序,而其发布的条件有所不同时,请
写信取得作者的许可。若为自由软件基金会享有著作权的软件,请写信至自由软
件基金会;我们有时会以例外方式予以处理。我们的决定取决于两项目标:确保
我们自由软件的所有衍生著作均维持在自由的状态,并广泛地促进软件的分享与
再利用。

无担保声明

\item 由于本程序是无偿授权,因此在法律许可范围內,本授权对本程序并不负
担保责任。非经书面声明,著作权人与/或其他提供程序之人,无论明示或默许,
均是依「现况」提供本程序而并无任何形式的担保责任,其包括但不限于,就适
售性以及特定目的的适用性为默示性担保。有关本程序品质与效能的全部风险均
由您承担。如本程序被证明有瑕疵,您应承担所有服务、修复或改正的费用。

\item 非经法律要求或书面同意,任何著作权人或任何可能依前述方式修改与/
或发布本程序者,对于您因为使用或不能使用本程序所造成的一般性、特殊性、
意外性或间接性损失,不负任何责任(包括但不限于,资料损失,资料执行不精
确,或应由您或第三人承担的损失,或本程序无法与其他程序运作等),即便前
述的著作权人或其他人已被告知该等损失的可能性时,也是一样。
\end{enumerate}

\section{您的新程序该如何采用这些条款?}
\label{sec:app:GPL:howToUse}
如果您开发了一个新程序,并且希望能够让它尽可能地被大众使用,达成此目的
的最好方式就是让它成为自由软件,任何人依这些条款规定都能就该软件再为发
布及修改。

为了做到这一点,请将以下声明附加到程序上。最安全的作法,是将声明放在每
份源码挡案的起始处,以有效传达无担保责任的讯息;且每份挡案至少应有「著
作权」列以及本份声明全文位置的提示。

用一行描述程序的名称与其用途简述著作权所有(C) 〈年份〉〈作者姓名〉

本程序为自由软件;您可依据自由软件基金会所发表的GNU通用公共授权条款规定,
就本程序再为发布与/或修改;无论您依据的是本授权的第二版或(您自行选择
的)任一日后发行的版本。

本程序是基于使用目的而加以发布,然而不负任何担保责任;亦无对适售性或特
定目的适用性所为的默示性担保。详情请参照GNU通用公共授权。

您应已收到附随于本程序的GNU通用公共授权的副本;如果没有,请写信至自由软
件基金会:59 Temple Place - Suite 330, Boston, Ma 02111-1307, USA。

同时附上如何以电子及书面信件与您联系的资料。

若程序是以交互方式运作时,请在交互式模式开始时,输出简短提示如下:

Gnomovision 第69版,著作权所有 (c) 年份 作者姓名Gnomovision不负担保责任,
欲知详情请键入'show w'。这是一个自由软件,欢迎您在特定条件下再发布本程
序;欲知详情请键入'show c'。

所假设的指令'show w'与'show c'应显示通用公共授权的相对应条款。当然,您
可以使用'show w'与'show c'以外的指令名称;甚至以鼠标键击或选菜单方式进
行-只要是合于您程序需要的方式都可以。

如有需要,您亦应取得您的雇主(若您的工作为程序设计師)或学校就本程序所
签署的「著作权放弃承諾书」。其范例如下,您只要修改姓名即可:

Yoyodyne公司,茲此放弃James Hacker所写之'Gnomovision'程序(该程序产出编
译器所需资讯)所有的著作权利益。

〈Ty Coon公司签章〉,1989年四月一日

Ty Coon公司,副总裁

本通用公共授权并不允许您将本程序并入专有程序中。若您的程序是一子程序函
数库时,您可能认为允许专有应用程序与该函式库相连结会更有帮助。若这是您
所想做的,请使用GNU函式库通用公共授权代替本授权。

%%% Local Variables:
%%% mode: latex
%%% TeX-master: "../SlackGuide"
%%% End: 
