
\chapter{Shell}
\label{chap:shell}

\begin{flushleft}
\rule[0mm]{\textwidth}{.1pt}
\end{flushleft}

在图形环境下,程序包含了窗口、滚动条、菜单等等,这些程序为用户提供了图
形接口。在命令行环境下,用户接口则是由shell提供的。shell的功能就是解释
命令,一般地,有了shell,一些东西就变得可用了。在登陆后(本章中涵盖了
该内容),用户会进入一个shell,并可以在shell中完成他们的工作。本章会对
shell作一个介绍,并介绍Linux用户中使用最多的一个shell---the Bourne
Again Shell(bash)。有关本章内容的更多细节,请参见见
\texttt{bash\textbf{(1)}}的man手册。

\section{用户}
\label{sec:shell:users}

\subsection{登陆}
\label{sec:shell:users:loggingIn}
那么,我们启动了系统,我们会看到诸如下面的内容:
\begin{Verbatim}[frame=single,commandchars=\\\{\}]
Welcome to Linux 2.4.18
Last login: Wed Jan
1 15:59:14 -0500 2005 on tty6.
darkstar login:
\end{Verbatim}
嗯……貌似没人说过应该怎么登陆。还有,darkstar是什么东东?darkstar只是
我们计算机的名字,而darkstar就是默认的名字。如果在安装过程中我们对计算
机名进行了设置(在netconfig中设置的),那么我们看到的就不是darkstar而
是我们自己设置的名字了。

至于登陆嘛……如果这是第一次启动,那么我们应该使用\texttt{root}登陆,
输入用户名后会提示输入密码,如果在安装过程中设置了\texttt{root}用户的
密码,那么这时候只要输入那个密码就可以了。如果当时没有设置密码,那么输
入回车即可。这就OK了---我们登陆了。

\subsection{root:超级用户}
\label{sec:shell:users:root}
OK,那么谁是或者说什么是\texttt{root}?在我们的系统中它又和用户帐号有
什么联系呢?

好的,在Unix及一些类似的操作系统(如Linux)的世界中,先有的是用户,然
后还是用户。之后我们会详细介绍,现在我们要知道\texttt{root}是一个用户,
且是在其它所有用户之前就存在的一个用户;\texttt{root}是全知的全能的,
没有用户可以违抗\texttt{root}。这是不允许发生的情况。\texttt{root}被称
为超级用户,它就是这么牛B。但最重要的是,\texttt{root}就是你自己。

是不是很嚣张?

不知道?那我们告诉你,是的,它就是很嚣张。但要注意的是,
\texttt{root}的设计使它能破坏一切它想破坏的东西。你可能希望先跳到第
\ref{sec:systemAdministration:usersAndGroups}节,学习如何创建一个普通用
户,并用普通用户登陆及工作。传统的用法是,只在绝对需要的情况下才切换到
超级用户,以此来保证破坏系统的最小可能性。

顺带一提,如果你先用其它用户登陆了,但想临时切换到root用户也是没问题的。
只要使用\texttt{su\textbf{(1)}}命令即可。它会要求你输入\texttt{root}用
户的密码,正确的话,它会让你拥有\texttt{root}权限,直到你用
\texttt{exit}或\texttt{logout}退出为止。当然,使用\texttt{su}命令可以
让你临时以任何用户登陆,只要知道密码即可。

\begin{quote}
  注意:\texttt{root}用户可以su到任意用户,而且不需要输入密码。
\end{quote}


\section{命令行}
\label{sec:shell:commandLine}

\subsection{运行程序}
\label{sec:shell:commandLine:runningPrograms}
不运行程序,几乎就做不成任何事,你可能会用你的电脑支撑一些东西;或着用
来挡着门,好让门一直开着;一些电脑在运行时还会发出可爱的嗡嗡响,但这并
不是我们所指的``做成事情''。我们可以相信,把计算机当作嗡嗡响的制门器并
不是计算机倍受喜爱的原因。

so,还记得Linux中几乎所有的东西都是用文件表示吗?好吧,这个规则对于程
序也适用。每个我们运行的(非shell内置的)程序都指向存在于某个地方的文
件。只要指定该文件的完整路径即可运行对应的程序。

例如,还记得上节中介绍的\texttt{su}命令吗?它实际位于\path{/bin}目录中:
输入\path{/bin/su}就可以正确运行它。

既然要输入完整路径,那为什么我们只输入\texttt{su}就能运行呢?毕竟我们
没有说它就在\path{/bin}目录下,它也可能在\path{/usr/local/share}目录下
对吧?那么shell是如何知道它的位置呢?这个问题的根源就在于环境变量
\texttt{PATH}中;多数的shell要么有环境变量\texttt{PATH},要么有一此和
\texttt{PATH}功能类似的变量。它包含一个列表,列表中的每一项就是我们可
能运行程序的目录。所以我们执行\texttt{su}命令时,shell会逐一搜索
\texttt{PATH}中的目录列表,在每个列表中寻找一个名为\texttt{su}的可执行
文件,遇到第一个\texttt{su}后,就运行它。在没有指定程序的完整路径时,
shell就会按如上的方法进行搜索,所以如果你碰到``Command not found''错误,
那就意味着你运行的程序所有的目录不在\texttt{PATH}中(当然,也可能是这
个程序圧根就不存在)。我们会在第
\ref{sec:shell:bash:environmentVariables}节中对环境变量进行深入介绍。

还记得``.''代表当前目录吗?所以如果你恰好就在\path{/bin}目录下,那么输
入\path{./su}就代表了完整的路径,所以也能正常运行。

\subsection{通配符匹配}
\label{sec:shell:commandLine:wildcard}
几乎所有的shell都会识别一些特定的字符,用来表示可以使用任意内容来替换
当前位置的内容。这些字符就被称作通配符(wildcard),最常用的两个就是
``*''和``?''。根据约定,?通常代表能匹配任意单个字符。例如,假设我们所
在的目录下有三个文件:\path{ex1.txt}、\path{ex2.txt}及\path{ext3.txt},
我们想把所有的文件复制到另一个文件夹下(使用在将在第
\ref{sec:handlingFilesAndDirectories:copyAndMove}节中介绍的\texttt{cp}
命令),假设是\path{/tmp}。那么一个直观的方法是使用命令
\begin{verbatim}
cp ex1.txt ex2.txt ex3.txt /tmp
\end{verbatim}
但这个命令要输入的东西太多了。我们可以使用上面介绍的?通配符,那么输入
\begin{verbatim}
cp ex?.txt /tmp
\end{verbatim}
就可以了。上面例子中,?会匹配字符``1''、``2''、``3'',这三个字符会轮流
替换上述命令中的?。

现在呢?是不是觉得要输的东西还是太多呢?是的。这是很令人震惊的,我们已
经有劳动法来保护我们,却还要做这么我的工作。幸运的是,我们还有一个通配
符——``*''。正如我们已经说过的,*可以匹配``任意多个任意的字符''。包括0
个字符。所以如果上面提到的三个文件是文件夹下仅有的三个文件,那么我们可
以使用如下命令就能一次性搞定。
\begin{verbatim}
cp * /tmp
\end{verbatim}
然而,如果文件夹下还有一个\path{ex.txt}文件和一个\path{hejaz.txt}文件,
我们希望复制\path{ex.txt}文件但不希望复制\path{hejaz.txt},那么只要使
用命令\verb|cp ex* /tmp|即可。

而\verb|cp ex?.txt /tmp|命令只会匹配最先提到的那三个文件,而
\path{ex.txt}中并没有额外的字符可以匹配?,所以不会被匹配。

另一个常用的通配符是中括号对``[ ]''。在进行匹配时,中括号中出现的任意
字符都可以替换到中括号所在的位置。听起来比较乱,举个例子:现在我们有一
个文件夹,文件夹中有下面8个文件:\path{a1,a2,a3,a4,aA,aB,aC,aD},现在
我们要列出以数字结尾的文件,那么使用``[ ]''就可以完成这个任务。
\begin{Verbatim}[frame=single, commandchars=\\\{\}]
darkstar~\% \textbf{ls a[1-4]}
a1 a2 a3 a4
\end{Verbatim}

但如果我们想要的只是\path{a1,a2}和\path{a4}呢?前面的例子中,我们用-来
表示从1到4的数字,我们也可以使用逗号来分隔单独的项。
\begin{Verbatim}[frame=single, commandchars=\\\{\}]
darkstar~\% \textbf{ls a[1,2,4]}
a1 a2 a4
\end{Verbatim}

我猜现在你在想的是,``如何匹配字符呢?''。Linux中是区分大小写的,这意
味着\texttt{a}与\texttt{A}是不同的字符,除了在我们脑袋之外,计算机是不
认为它们之间有什么关联的。要注意的是大写的字母在字母表中位于小写字母的
前面。所以\texttt{A}和\texttt{B}出现在\texttt{a}各\texttt{b}的前面。继
续之前的例子,如果我们想显示的文件名是\path{a1}及\path{A1},也可以快速
地用``[ ]''实现。
\begin{Verbatim}[frame=single, commandchars=\\\{\}]
darkstar~\% \textbf{ls [A,a]1}
A1 a1
\end{Verbatim}

注意,如果我们不是用逗号而是用横杠(`-'),那么得到的就是错误的结果了

\begin{Verbatim}[frame=single, commandchars=\\\{\}]
darkstar~\% \textbf{ls [A-a]1}
A1 B1 C1 D1 a1
\end{Verbatim}

我们也可以结合使用横杠和逗号来表示一个串。

\begin{Verbatim}[frame=single, commandchars=\\\{\}]
darkstar~\% \textbf{ls [A,a-d]}
A1 a1 b1 c1 d1
\end{Verbatim}


\subsection{重定向输入输出及管道}
\label{sec:shell:commandLine:redirectionAndPiping}
(接下来的东西就相当有意思了)
\begin{Verbatim}[frame=single, commandchars=\\\{\}]
darkstar~\% \textbf{ps > blargh}
\end{Verbatim}

知道这是什么吗?这是我在运行\texttt{ps}命令来查看当前运行的进程。
\texttt{ps}命令会在第\ref{sec:processControl:ps}节中进行介绍,这不是有
意思的部分,有意思的是\texttt{> blargh},它的意思是将\texttt{ps}命令的
输出写到一个名为\texttt{blargh}文件中。但先等等,下面的东西更有意思。
\begin{Verbatim}[frame=single, commandchars=\\\{\}]
darkstar~\% \textbf{ps | less}
\end{Verbatim}

这个命令将\texttt{ps}命令的输出通过管道送给\texttt{less},这样我就可以
在我想滚动的时候滚动屏幕了。

\begin{Verbatim}[frame=single, commandchars=\\\{\}]
darkstar~\% \textbf{ps >> blargh}
\end{Verbatim}

这是第三个最常用的重定向,它的作用与``$>$''一致,只是,当
\texttt{blargh}文件存在时,``$>>$''会将\texttt{ps}命令的输出追加到
\texttt{blargh}文件中;如果不存在,它的作用与``$>$''一致,会先创建文件,
再将内容写入文件。(``$>$''会擦除\texttt{blargh}文件的所有内容。)

还有一个重定向符:``<'',它的作用是从该符号之后指定的位置读取输入,相
比之下用得不多。
\begin{Verbatim}[frame=single, commandchars=\\\{\}]
darkstar~\% \textbf{fromdos < dosfile.txt > unixfile.txt}
\end{Verbatim}

如果我们学会使用管道,很多东西就变得很有意思:

\begin{Verbatim}[frame=single, commandchars=\\\{\}]
darkstar~\% \textbf{ps | tac >> blargh}
\end{Verbatim}
上面这个命令会先运行\texttt{ps},将输出的每一行反转后追加到文件
\texttt{blargh}中。你可以按自己的需要将一些命令组合进来,只要记得它们
是从左往右解释的就行了。

关于重定向的更多知识,请参见\texttt{bash\textbf{(1)}}的man手册。

\section{The Bourn Again Shell (bash)}
\label{sec:shell:bash}

\subsection{环境变量}
\label{sec:shell:bash:environmentVariables}

Linux系统是个复杂的畜牲,在运行的时候需要记录很多东西,在与其它程序正
常交互的时候需要用到很多的细节(有些你可能从来都没意识到)。没有人想在
运行程序的时候为程序传递一堆参数,告诉程序要用什么样的终端、计算机的主
机名是什么,要使用什么样的提示信息……

就好比是一个自动复制的机器一样,我们把它叫作环境变量。所有的Shell都用环
境变量来记录信息,来方便用户的使用。所谓的环境变量只是一堆信息的缩写,
用户可能希望记录这些信息以便之后使用。例如,环境变量PS1告诉Shell如何显
示提示符,另外一些变量可能是用来告诉程序如何运行的。这里,我们会对bash
提供的处理环境变量的一些命令进行简要介绍。

\texttt{set}在单独使用时会显示当前定义的所有环境变量。就像多数bash内置
的命令一样,它还能完成一些其它的任务(加参数),关于这些内容,请自己查
阅\texttt{bash\textbf{(1)}}的相关文档。下面的例子中就是在作者机子上单
独运行\texttt{set}使得得到结果的一个摘录。注意到我们之前提到的
\texttt{PATH}环境变量。在\texttt{PATH}变量的目录中的程序只要输入文件名
就可以运行了,而不需要完整路径。

\begin{Verbatim}[frame=single, commandchars=\\\{\}]
darkstar~\% \textbf{set}
PATH=/usr/local/lib/qt/bin:/usr/local/bin:/usr/bin:/bin:/usr/X11R6/bin:
/usr/openwin/bin:/usr/games:.:/usr/local/ssh2/bin:/usr/local/ssh1/bin:
/usr/share/texmf/bin:/usr/local/sbin:/usr/sbin:/home/logan/bin
PIPESTATUS=([0]="0")
PPID=4978
PS1='\textbackslash{}h:\textbackslash{}w\textbackslash{}\$ '
PS2='> '
PS4='+ '
PWD=/home/logan
QTDIR=/usr/local/lib/qt
REMOTEHOST=ninja.tdn
SHELL=/bin/bash
\end{Verbatim}

\begin{Verbatim}[frame=single, commandchars=\\\{\}]
darkstar~\% \textbf{unset VARIABLE}
\end{Verbatim}

\texttt{unset}会移除环境变量,包括环境变量的值及环境变量本身
,\texttt{bash}会忘记该变量的存在(不要害怕,除非是你在某个shell会话中
单独定义的变量,否则其它会话里变量还是会重新定义的,那就是默认的一些变
量)。

\begin{Verbatim}[frame=single, commandchars=\\\{\}]
darkstar~\% \textbf{export VARIABLE=some\_value}
\end{Verbatim}

使用\texttt{export}真的很方便。它可以为某个环境变量\texttt{VARIABLE}赋
值为``some\_value'',如果环境变量\texttt{VARIABLE}之前不存在,那么执行
完上述命令后它就存在了。如果环境变量\texttt{VARIABLE}之前有其它值,那
么现在也被删除,替换为新的值。当然,如果我们只是想在\texttt{PATH}中加
入一个新的文件夹,那么这个特性就不太适合了,这个情况下,我们可能会采用
如下命令:
\begin{Verbatim}[frame=single, commandchars=\\\{\}]
darkstar~\% \textbf{export PATH=\$PATH:/some/new/directory}
\end{Verbatim}

注意\texttt{\$PATH}这里的用法:当我们希望\texttt{bash}解释一个变量时
(即得到它原有的值),在变量名前加上\$符即可。例如\verb|echo \$PATH|就
会将\texttt{PATH}中的值显示出来,如我的例子中:
\begin{Verbatim}[frame=single, commandchars=\\\{\}]
darkstar~\% \textbf{echo \$PATH}
/usr/local/lib/qt/bin:/usr/local/bin:/usr/bin:/bin:/usr/X11R6/bin:
/usr/openwin/bin:/usr/games:.:/usr/local/ssh2/bin:/usr/local/ssh1/bin:
/usr/share/texmf/bin:/usr/local/sbin:/usr/sbin:/home/logan/bin
\end{Verbatim}


\subsection{Tab补全}
\label{sec:shell:bash:tabCompletion}
(这个东西又是很有意思的)
\begin{enumerate}
\item 命令行接口意味着要输入很多内容。
\item 输入是一项工作
\item 没人喜欢工作
\end{enumerate}
由第3点和第2点,我们可以得到第4点:没人喜欢输入。幸运的是,
\texttt{bash}作了一些设计,防止第5点(没有喜欢命令行接口 )的发生。

那么我们会问:\texttt{bash}是怎么完成这项壮举的呢?除了之前我们介绍的
通配符匹配,\texttt{bash}还有一个特性:tab补全。

Tab补全的工作原理如下:我们输入一个文件名,它可能在你的\texttt{PATH}中,
也可能由我们手工输入完整路径。不论是哪种情况,我们要做的只是输入足够的
内容使我们能唯一定位某个文件,之后按下tab键就可以了。\texttt{bash}会为
我们完成剩下的输入。

当然,还得用例子来解释。假设\path{/usr/src}目录下有两个子文件夹:
\path{/usr/src/linux}和\path{/usr/src/sendmail}。现在我们想知道
\path{/usr/src/linux}目录下有什么,所以我只要输入\texttt{ls
  /usr/src/l},之后按TAB键,\texttt{bash}就会补全得到\texttt{ls
  /usr/src/linux}。

现在,假设我们有两个目录\path{/usr/src/linux}和
\path{/usr/src/linux-old},如果我们输入\path{/usr/src/l}后键入TAB,
\texttt{bash}会为我们补全尽可能多的内容,因此我们会得到
\path{/usr/src/linux},当然,到这点我就可以停下了,或者我也可以再敲下
TAB键,\texttt{bash}会显示一张列表,列出与我当前输入内容匹配的所有目录
的列表。

从今往后,我们就不要输入这么多内容了(也因此,人们喜爱命令行接口)。是
吧,我说过这个东西很有意思的。


\section{虚拟终端}
\label{sec:shell:virtualTerminals}
想像这样一种情况,你在做着一件事,但突然决定要先做另一件事。现在我们必
须先停下当前的工作,并切换到准备完成的任务。但还记得Linux是多用户系统
吗?也就是我们想同时登陆多少次都可以的对吗?所以为什么我们只能一次做一
件事呢?

的确没必要,我们不可能都为同一个机器配备好几个键盘、鼠标,更关键的是我
们中的多数人也不想要这么多东西。这就很明了了,增加硬件并不是我们要的答
案。那么就剩下软件了,Linux已经在这方面做了一定的工作,为我们提供了
``虚拟终端(virtual terminals)''或简称``VTs''。

按下Alt及功能键(F1~F12),就能在虚拟终端中进行切换,每个功能键对应一
个虚拟终端。Slackware默认提供了6个VT,按下Alt+F2就可以切换到第二个虚拟
终端中,Alt+F3对应第三个,以此类推。

剩下的功能键为X会话保留,每个X会话使用自己的VT,从第7个(Alt+F7)往上。
在X中,Alt+Fn键会被捕捉的,因此,在X中使用Ctrl+Alt+Fn进行切换。所以如
果我们在X中,想切换回文本终端(而不退出X会话),那么如Ctrl+Alt+F3会切
换到第三个终端中。(如果使用X会话,那么Alt+F7会再切回X。)

\subsection{Screen}
\label{sec:shell:virtualTerminals:screen}
但如果在没有虚拟终端的情况下呢?怎么办?幸运的是,Slackware还包括了一
个很好的屏幕管理器,而名字就叫作\texttt{screen}。\texttt{screen}是一个
终端模拟器,有着与虚拟终端近似的功能。执行\texttt{screen}会闪出一个简
短的介绍,之后就进入一个终端。与传统虚拟终端不同的是,\texttt{screen}
有自己的命令。所有的\texttt{screen}命令都要先按下Ctrl+A组合键。例如,
Ctrl+A+C的作用是创建一个新的终端会话,Ctrl+A+N的作用就是切换到下一个终
端,对应的Ctrl+A+P会切换到上一个终端中。

\texttt{screen}还提供了从\texttt{screen}会话中分离及重新附加到某一会话
中的功能,这对于通过\texttt{ssh}及\texttt{telnet}(之后会介绍)进行的
远程连接来说异常有用。执行\verb|screen -r|命令会列出当前运行的可以重新
附加到的会话。

\begin{Verbatim}[frame=single, commandchars=\\\{\}]
darkstar~\% \textbf{screen -r}
There are several suitable screens on:
     1212.pts-1.redtail (Detached)
     1195.pts-1.redtail (Detached)
     1225.pts-1.redtail (Detached)
     17146.pts-1.sanctuary (Dead ???)
Remove dead screens with ’screen -wipe’.
Type "screen [-d] -r [pid.]tty.host" to resume one of them.
\end{Verbatim}
运行\verb|screen -r 1212|会重新附加到表中的第一个screen。我之前提到过
它对于远程连接的优势。如果我们使用\texttt{ssh}登陆到一个远程的
Slackware服务器上,现在我们的连接可能由于本地机器断电等原因而断开,那
么无论我当前在做什么工作,都会瞬间毁灭,这可能对服务器也很伤。如果使用
\texttt{screen},那么在连接断开时,它会将当前的会话从screen中分离开,
一旦连接恢复,就可以重新附加到之前的screen会话中,并从断开处继续工作。


%%% Local Variables: 
%%% mode: latex
%%% TeX-master: "../SlackGuide"
%%% End: 
