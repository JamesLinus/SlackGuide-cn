\chapter*{\markboth{写在前面}{写在前面}写在前面}
\label{chap:preface}

\addcontentsline{toc}{chapter}{写在前面}

\begin{flushleft}
\rule[0mm]{\textwidth}{.1pt} % a separator
\end{flushleft}

\section*{关于译本}
\label{sec:preface:about}
本译本是由\LaTeX{}写成的,源码及PDF可在
\url{https://github.com/lotabout/SlackGuide-cn.git}中自由下载。

如果你发现其中的错误,或有什么建议(如封面、格式等等),可以给直接给笔
者发邮件\texttt{lotabout@gmail.com},也可以考虑直接修改到github中,谢谢。
本书的所有更新也会在github中给出。

关于版权,你可以自由地复制和打印本书,但在用于任何商业目的之前必需联系
笔者。

\section*{无心插柳}
\label{sec:preface:purpose}

笔者是从Slackware 13.1开始接触Slackware的,光盘中就附有Slackbook,一本
很好的官方指南。但由于年代久远(第二版是2005年发行的,写作时是2012年),
Slackware已经发生了很大的改变,其中的很多内容已不再适用。在网上发现
2009年是就已经出了Slackbook 3.0的beta版,可正式版仍迟迟不出。一方面为
了学习Slackware,另一方面为了为Slackware尽一点绵薄之力,遂决定翻译
Slackbook。

笔者为Slackbook 3.0的beta版翻译了六章,由于beta版的内容实在过于简单,
与Slackbook 2正式版实有天壤之别,最终决定放弃。但Slackbook 2已经过时,
实用价值大打折扣,遂决定在Slackbook 2基础之上,以翻译为主要途径,对其
中过时的内容加以更新,提供一本好用的Slackware中文指南。

书中会尽量以笔者的使用经验对相关的内容进行陈述,尽量使用第一人称。
由于很多部分是翻译而来,其中会夹杂着以原作者为第一人称的描述。请见谅。
另外,笔者水平有限,多数翻译都采用直译,如有不恰当之处,敬请批评指正。

\section*{如何使用本书}
\label{sec:preface:howToUse}
本书只是对Slackware(及一般的Linux)的基础知识进行简要的陈述,让你了解
Slackware的基本方面。而对多数的知识点只是浅尝辄止,没有深入介绍。

首先,如果你是一个完全的新手,可以通过本书来了解Slackware的一般操作,了
解Slackware包含的内容,之后再通过查阅相关的man手册或其它文档、教程来进
一步学习相关的内容,最终才能掌握Slackware。另外,你还可以根据本书来检验
你对Slackware的掌握程度。

另一方面,本书可以作为一个Slackware的快速参考,查找那些一时间想不起来
的知识。

当然,如果你想到其它的妙用,请及时告知。


\section*{目标受众}
\label{sec:preface:intendedAudience}
Slackware Linux 操作系统作为一个功能强大的平台,是为基于Intel处理器的计
算机设计的。它的设计目标是成为一个稳定、安全、实用的高端服务器以及功能
强劲的工作站。

本书的初衷是带你走进Slackware Linux操作系统。这并不意味着本书涵盖了
Slackware的方方面面,而是为你展示它能做些什么,并且教会你使用该
系统的基础知识。

如果你是Slackware的老手,那么我们希望本书可以作为参考手册使用。 当然,
我们也希望,如果有人向你问起Slackware,你可以向他们介绍介绍本书。
Slackware,谁用谁知道。

当然,这本书不可能像小说一样吸引人,我们会尽我们所能地写得更有趣。写得
好的话说不定还会有人请我们写剧本?当然,我们最希望的是你能从中学习到你
认为有用的知识。

现在,让好戏开锣吧!

\section*{为什么写本新的?}
\label{sec:preface:whyANewSlackwareBook}
在现存的Linux发行版中,Slackware Linux算得上是爷爷辈的,但这并不
意味着它与时代脱节。Slackware的确尝试过保持纯正的Unix风格,但还是没能逃
过时代的进步的历程。子系统不一样了,窗口管理器变来变去,人们又发明了新的方
法来管理复杂的现代操作系统。我们确实反对为了改变而改变,但当事物进化了,相
应的文档就变得陈旧了,一切都是无法避免的——包括书籍。


\section*{本书中的一些约定}
\label{sec:preface:conventionsUsedInThisBook}

为了保持一致性,同时方便大家阅读,全书遵守一些约定。

\subsection*{印刷上的约定}
\label{sec:preface:conventions:typographic}

\begin{description}
\item[\textit{Italic}] 斜体字\textit{Italic}用于表示命令、强调、及技术名词第一次出现的地
  方。
\item[\texttt{Monospace}] 等宽字体\texttt{monospaced}用于表示错误信息、
  命令、环境变量、端口名、主机名、用户名、组名、设备名、变量及代码段。
\item[\textbf{Bold}] 粗体字用于表示例子中的用户输入。
\end{description}

\subsection*{用户输入}
\label{sec:preface:conventions:userInput}

输入用粗体表示,用以区别其它的字符。至于键组合,我们用`+'来连接各个字
符以表示要同时按下,例如:
\begin{Verbatim}[frame=single]
Ctrl+Alt+Del
\end{Verbatim}
意味着我们必须同时按下\textit{Ctrl},\textit{Alt},以及\textit{Del}键。

有时候是要先后键入的,我们用逗号来分隔它们,例如:
\begin{Verbatim}[frame=single]
Ctrl+X, Ctrl+S
\end{Verbatim}
就代表我们要先同时输入\textit{Ctrl}和\textit{X}键,接着同时输入
\textit{Ctrl}键和\textit{S}键。

\subsection*{例子}
\label{sec:preface:conventions:examples}

以\texttt{E:\textbackslash{}\textgreater}开始的例子代表这是一个
MS-DOS\textregistered{}命令。除非特别说明,这些命令都是在
\textbf{Microsoft}\textregistered{}
\textbf{Windows}\textregistered{}``命令提示符''之下运行的。
\begin{Verbatim}[frame=single]
D:\> rawrite a: bare.i
\end{Verbatim}

以\texttt{\#}开始的例子代表执行一个命令时必须使用Slackware中超级用户的
权限执行。你可以直接以\texttt{root}登陆,也可以以普通用户登陆,之后再
用\texttt{su(1)}来获得超级用户的权限。
\begin{Verbatim}[frame=single]
# dd if=bare.i of=/dev/fd0
\end{Verbatim}

以\texttt{\%}开始的例子代表执行命令时只须使用普通用户权限。除非特别说
明,我们将使用C-shell语法来设置环境变量以及其它的shell命令。
\begin{Verbatim}[frame=single]
% top
\end{Verbatim}

\section*{致谢}
\label{sec:preface:acknowledgments}

这个项目是由许多人花费几个月的贡献累积而成的。仅凭我一已之力,是不可能
凭空创造出的。我们要感谢许许多多的人,感谢他们无私的奉献:Keith Keller
写了无线网络方面的内容;Joost Kremers单枪匹马完成了emacs这一章;Simon
Williams完成了"安全"这章;Jurgen Phillippaerts完成"基础网络命令"的内容;
以及Cibao Cu Ali G Colibri带来的启发和无穷的能量\footnote{原文为 a
good kick in the pants. 译者对这些幽默不是很理解。这个英文解释是forceful encouragement to do something}。还有数不清的人为
我们提出了建议及文章的修正。下面是一个完整的列表:Jacob Anhoej, John
Yast, Sally Welch, Morgan Landry, 及Charlie Law。同时,我也要感谢Keith
Keller为这个项目管理邮件列表,感谢Carl Inglis最开始为我们托管网站。还有,虽然放到最后,但却是最重要的,我要感谢Patrick J. Volkerding创建了Slackware Linux,以及
David Cantrell, Logan Johnson及Chris Lumens,是他们完成了Slackware
Linux Essentials的第一版。如果没有他们打好的地基,本书的一切都不可能发生。还
有很多人,虽然在这里没有列出他们的名字,但都以各种方式为这个项目做出了贡献。希望
他们能原谅我记忆不好。

Alan Hicks。 2005年5月。




%%% Local Variables: 
%%% mode: latex
%%% TeX-master: "../SlackGuide"
%%% End: 
