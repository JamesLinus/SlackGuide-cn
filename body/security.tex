
\chapter{安全性}
\label{chap:security}

\begin{flushleft}
\rule[0mm]{\textwidth}{.1pt}
\end{flushleft}

任何系统中,安全性都是极为重要的。安全性意味着可以防止别人对我们的机器
攻击,同时保护敏感数据。本章的内容就是如何保护Slackware的安全,防止恶
意脚本、破解、及流氓程序。请记住,这只是保护一个系统的开始,安全是一个
过程而不是一个状态。

\section{关闭服务}
\label{sec:security:disablingServices}
安装完Slackware后的第一件事应该是关闭那些你不需要的服务。还记得这个说
法吗?服务越多则安全性越低。每个服务都可能增加安全隐患,所以我们要尽可
能少开启服务(即只开启我们需要的服务)。服务主要从两个地方开
启——\texttt{inetd}及启动脚本。

\subsection{inetd开启的服务}
\label{sec:security:disablingServices:inetd}
Slackware的许多守护进程都是由\texttt{inetd}(8)开启的。\texttt{inetd}本
身就是一个守护进程,有许多服务被设置为由\texttt{inetd}启动,而
\texttt{inetd}就负责监听这些服务用到的端口,当接收到一个连接请求时,
\texttt{inetd}就创建该服务对应守护进程的一个实例。由\texttt{inetd}启动
的守护进程可以通过注释掉\path{/etc/inetd.conf}文件中的相应行来关闭。用
你习惯的文本编辑器(如\texttt{vi})打开该文件,你会看到类似下面的行:
\begin{Verbatim}[frame=single, commandchars=\\\{\}]
telnet stream tcp nowait root /usr/sbin/tcpd in.telnetd
\end{Verbatim}
我们可以关闭该服务,以及其它不需要的服务,只要将这行注释掉(在行首加上
\#号)即可。注释后如下所示:
\begin{Verbatim}[frame=single, commandchars=\\\{\}]
#telnet stream tcp nowait root /usr/sbin/tcpd in.telnetd
\end{Verbatim}

在\texttt{inetd}重启之后,这些服务就会被关闭,可以通过如下命令来重启
\texttt{inetd}:
\begin{Verbatim}[frame=single, commandchars=\\\{\}]
# \textbf{kill -HUP \$(cat /var/run/inetd.pid)}
或
# \textbf{/etc/rc.d/rc.inetd restart}
\end{Verbatim}

\subsection{启动脚本开启的服务}
\label{sec:security:disablingServices:init}
剩下的一些服务是由\path{/etc/rc.d/}目录下的启动脚本开启的。这些服务可
以通过两种方法禁用。第一种是去除相应脚本的可执行权限,第二种则是在初始
化脚本中注释掉对应的行。

例如,SSH服务是由单独的初始化脚本\path{/etc/rc.d/rc.sshd}开启的,我们
可以使用如下命令禁用它:
\begin{Verbatim}[frame=single, commandchars=\\\{\}]
# \textbf{chmod -x /etc/rc.d/rc.sshd}
\end{Verbatim}

而对于没有单独启动脚本的一些服务,我们就必须在相应的启动脚本中注释相应
的行,才能禁用它们。例如:portmap守护进程是在\path{/etc/rc.d/rc.inet2}
脚本中的下列行中启动的\footnote{本服务在Slackware 13.37中不再是由此方
  法启动的。}:
\begin{Verbatim}[frame=single, commandchars=\\\{\}]
# This must be running in order to mount NFS volumes.
# Start the RPC portmapper:
if [ -x /sbin/rpc.portmap ]; then
  echo "Starting RPC portmapper: /sbin/rpc.portmap"
  /sbin/rpc.portmap
fi
# Done starting the RPC portmapper.
\end{Verbatim}
我们可以通过在相应语句前加上\texttt{\#}号注释来禁用它。如:
\begin{Verbatim}[frame=single, commandchars=\\\{\}]
# This must be running in order to mount NFS volumes.
# Start the RPC portmapper:
#if [ -x /sbin/rpc.portmap ]; then
# echo "Starting RPC portmapper: /sbin/rpc.portmap"
# /sbin/rpc.portmap
#fi
# Done starting the RPC portmapper.
\end{Verbatim}

做完以上的工作,只有在下次重启或切换到运行级别3或4时才会生效。我们可以
在终端中输入下面的命令来进行切换(切换到运行级别1后要重新登陆):
\begin{Verbatim}[frame=single, commandchars=\\\{\}]
# \textbf{telinit 1}
# \textbf{telinit 3}
\end{Verbatim}


\section{主机访问控制}
\label{sec:security:hostAccessControl}

\subsection{iptables}
\label{sec:security:hostAccessControl:iptables}
\texttt{iptables}是2.4及以上版本的内核中的包过滤配置程序。Slackware第
一个使用2.4内核(准确的说是2.4.5)的版本是8.0,而在Slackware 8.1中默认
使用它。本书中只对它的基本用法进行介绍,全面的介绍请参考
\url{http://www.netfilter.org}。这些iptables命令可以写入
\path{/etc/rc.d/rc.firewall}文件\footnote{Slackware 13.37中默认没有该
  文件,我们可以手工创建,它会由\texttt{rc.inet2}进行调用。},当然,要
默认启动的话,还需要为其设置可执行权限。注意,错误的\texttt{iptables}
命令可能会锁住机器,连自己都进不去。因此,除非你对自己100\%自信,否则
最少保证自己能有个本地访问的权限。

多数人会做的第一件事就是用DROP为每个内部链路设置默认策略:
\begin{Verbatim}[frame=single, commandchars=\\\{\}]
# \textbf{iptables -P INPUT DROP}
# \textbf{iptables -P FORWARD DROP}
\end{Verbatim}
这个命令的作用是阻止一切,现在我们可以开始逐渐开放权限。我们第一件要做
的事就是允许传递已建立会话(sessioin)的数据:
\begin{Verbatim}[frame=single, commandchars=\\\{\}]
# \textbf{iptables -A INPUT -m state --state ESTABLISHED,RELATED -j ACCEPT}
\end{Verbatim}

为了保证那些使用环回地址进行通信的程序能正常工作,一般添加下面的规则是
比较明智的:
\begin{Verbatim}[frame=single, commandchars=\\\{\}]
# \textbf{iptables -A INPUT -s 127.0.0.0/8 -d 127.0.0.0/8 -i lo -j ACCEPT}
\end{Verbatim}

这个规则允许了所有通过回环接口(lo)传递给127.0.0.0/8(即
127.0.0.0-127.255.255.255)的数据包。创建规则的时候越具体越好,这样可
以防止不小心给潜在的威胁留下漏洞。但每条规则允许的东西少了,也就意味着
我们要创建更多的规则。

接下来我们要做的就是允许访问我们机器上运行的一些特定服务。例如,我们想
在我们的机器上运行一个网页服务器,我们就要用类似下面的规则:
\begin{Verbatim}[frame=single, commandchars=\\\{\}]
# \textbf{iptables -A INPUT -p tcp --dport 80 -i ppp0 -j ACCEPT}
\end{Verbatim}
这条规则允许其它机器通过\texttt{ppp0}接口访问我们机器的80号端口。我们
也可能只想让一部分机器访问这些服务。下面的规则则只允许从
\url{64.57.102.34}访问我们的网页服务器:
\begin{Verbatim}[frame=single, commandchars=\\\{\}]
# \textbf{iptables -A INPUT -p tcp -s 64.57.102.34 --dport 80 -i ppp0
  -j ACCEPT}
\end{Verbatim}

如果想用于检测一些网络问题,允许ICMP通信是比较有用的。我们可以通过类似
下面的规则来完成这个任务:
\begin{Verbatim}[frame=single, commandchars=\\\{\}]
# \textbf{iptables -A INPUT -p icmp -j ACCEPT}
\end{Verbatim}

多数人可能还想在它们作为网关的机器建立网络地址转换(NAT)服务,那么它
所有的内网就可以通过该机器访问网络了。我们可以使用下面的规则:
\begin{Verbatim}[frame=single, commandchars=\\\{\}]
# \textbf{iptables -t nat -A POSTROUTING -o ppp0 -j MASQUERADE}
\end{Verbatim}

当然,我们还需要启用IP转发功能。如果想临时开启,那么使用下面的命令:
\begin{Verbatim}[frame=single, commandchars=\\\{\}]
  # \textbf{echo 1 > /proc/sys/net/ipv4/ip_forward}
\end{Verbatim}

如果想在每次开机时都启动IP转发功能,那么可以为
\path{/etc/rc.d/rc.ip_forward}添加可执行权限:
\begin{Verbatim}[frame=single, commandchars=\\\{\}]
# \textbf{chmod +x /etc/rc.d/rc.ip_forward}
\end{Verbatim}

更多关于NAT的信息,请参见NAT
HOWTO\footnote{\url{http://www.netfilter.org/documentation/HOWTO/NAT-HOWTO.txt}}

\subsection{tcpwrappers}
\label{sec:security:hostAccessControl:tcpwrappers}

\texttt{tcpwrapper}是在软件层执行访问控制而非IP层。当IP层的访问控制
(即Netfilter)出现异常时,它可以提供额外的保护层。例如,如果你重新编
译了内核,忘了开启iptables支持,则IP层的访问控制就会失效,这时候
\texttt{tcpwrappers}仍旧可以对系统进行保护。

由\texttt{tcpwrapper}提供的保护服务可以由\path{/etc/hosts.allow}及
\path{/etc/hosts.deny}文件进行控制。

多数人的\path{/etc/hosts.deny}中只有一行,用来默认阻止所有的访问,这行
看起来可能像下面这样:
\begin{Verbatim}[frame=single, commandchars=\\\{\}]
ALL : ALL
\end{Verbatim}

当我们用上面的方法默认阻止所有访问时,我们就可以指定那些我们允许的主机、
域名或一个范围的IP,赋予它们访问的权限。这可以通过
\path{/etc/hosts.allow}文件达成,它的语法和前面一致。

例如,许多人会允许所有对回环地址的访问,则可以在
\path{/etc/hosts.allow}文件中添加如下规则:
\begin{Verbatim}[frame=single, commandchars=\\\{\}]
ALL : 127.0.0.1
\end{Verbatim}

而要允许从\url{192.168.0.0/24}地址段的机器访问SSHd,则可以使用下列任意
规则之一:
\begin{Verbatim}[frame=single, commandchars=\\\{\}]
sshd : 192.168.0.0/24
ssh  : 192.168.0.
\end{Verbatim}

我们还能限定某个域名对主机的访问权限。这可以用下面的规则实现(注意,这
个方法依赖于DNS的反向解释,所以除非DNS服务器是可信赖的,否则效果不好,
所以我们不建议你在因特网上使用该规则。):
\begin{Verbatim}[frame=single, commandchars=\\\{\}]
sshd : .slackware.com
\end{Verbatim}

\section{保持最新状态}
\label{sec:security:keepingCurrent}

\subsection{slackware-security邮件列表}
\label{sec:security:keepingCurrent:mailingList}
当某个安全问题影响到Slacwkare时,Slackware官方会对所有订阅了
\url{slackware-security@slackware.com}邮件列表的用户发送一封邮件。而除
了在\path{/extra}和\path{/pasture}中的软件,Slackware中其它软件的报告
会被发送给那些志愿进行修改的志愿者。这些安全声明通常会包括软件包更新后
的信息,如果有一些解决方法的话,也会包含其中。

如何订阅这些邮件已经在第\ref{sec:help:onlineHelp:mailingLists}节中介绍过了。

\subsection{/pasture目录}
\label{sec:security:keepingCurrent:pasture}
当一个更新后的软件包在一个新的Slackware版本发行的时候(如果它已经包含在
Slackware之前的版本中,那么通常意味着新的版本只是修复了某些安全问题),
那么Slackware会把它放在\path{/pastures}目录。具体的目录根据你用的镜像
站点不同而不同。

在安装这些软件包之前,最好先检查它们的\texttt{md5sum}。
\texttt{md5sum}(1)是一个命令行工具,它可以方便地为一个文件生成一个唯一
的散列值。如果一个文件的某一位发生了改变,它的md5sum值也会有很大的不同。
\begin{Verbatim}[frame=single, commandchars=\\\{\}]
\% \textbf{md5sum package-<ver>-<arch>-<rev>.tgz}
6341417aa1c025448b53073a1f1d287d package-<ver>-<arch>-<rev>.tgz
\end{Verbatim}

得到md5sum值后,我们可以与\path{CHECKSUMS.md5}文件中的对应包的md5sum值
进行比较。\path{CHECKSUMS.md5}文件位于\texttt{slackware-\$VERSION}目录
下(也在\path{/patches}目录下)或都在 \texttt{slackware-security} 邮件列
表的邮件中。

如果你有一个包含了各个包的md5sum值的文件,那么可以使用\texttt{md5sum}
的[\texttt{-c}]参数。
\begin{Verbatim}[frame=single, commandchars=\\\{\}]
# \textbf{md5sum -c CHECKSUMS.md5}
./ANNOUNCE.10_0: OK
./BOOTING.TXT: OK
./COPYING: OK
./COPYRIGHT.TXT: OK
./CRYPTO_NOTICE.TXT: OK
./ChangeLog.txt: OK
./FAQ.TXT: FAILED
\end{Verbatim}

正如我们看到的,\texttt{md5sum}检查正确的文件都以``OK''列出,而MD5值不
通过的则标记为``FAILED''。(这个解释是在侮辱您的智商,为什么你还能忍得
了?)



%%% Local Variables: 
%%% mode: latex
%%% TeX-master: "../SlackGuide"
%%% End: 
