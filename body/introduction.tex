
\chapter{Slackware简介}
\label{chap:introduction}

\rule[0mm]{\textwidth}{0.1pt}

\section{什么是Linux?}
\label{sec:intro:whatIsLinux}
Linux是一个操作系统的内核,是Linus Torvalds于1991年作为一个私人的项目
完成的。最初他开始这个项目的目的只是为了不花钱就能得到一个基于Unix的操
作系统。另外,他还想学习386处理器的细节。Linux是在遵守通用公共许可(见
第\ref{sec:intro:openSourceandFreeSoftware}节及附录\ref{chap:app:GPL}中对该许可的介绍)
下发行的,对于公众它是免费的,且任何人都能自由地学习并进行改善。现在,
Linux已经成为操作系统市场上的一个重要的竞争者。人们已经把它移植到了许
多架构上运行,包括HP/Compaq的Alpha,Sun公司的SPARC及UltraSPARC,以及
Motorola的PowerPC芯片(在如苹果公司的Macintosh及IBM的RS/6000等计算机上
运行)。当然世界上,即使没有上千,也有成百的程序员在开发Linux。在Linux
上可以运行如Sendmail、Apache及BIND等软件,这些都是当前流行的用来搭建英
特网服务器的软件。要时刻记得,Linux这个词代表的是内核——一个操作系统的
核心。这个核心的作用是控制计算机的处理器、内存、硬件驱动及外围设备。这
也是Linux的所有功能:控制系统的运转并保证程序正常运行。一些公司或个人
将Linux内核与一些程序绑定在一起构成一个操作系统。我们将每一个绑定称作
一个Linux发行版。

\subsection{GNU的世界}
\label{sec:intro:GNU}
Linux 内核项目最早是由Linux Torvalds在1991年时独自努力实现的,但正如牛
顿说过的“我看得远,是因为站在巨人的肩上”一样,正当Linus Torvalds准备
开发内核时,自由软件基金会已经有了协作软件的想法了。他们将他们的想法命
名为GNU,GNU是一个递归的缩写词,全称为“GNU's Not Unix”。GNU的软件从
Linux元年1月1日\footnote{原文为`day 1'}开始运行在Linux内核上。人们用他
们编译器\texttt{gcc}来编译内核。直至今天,许多GNU工作仍是每个主流Linux
发行版的基础,从\texttt{gcc}到\texttt{gnutar}。因此,许多自由软件基金
会的支持者都坚持认为他们所做的工作有着不小于Linux内核的功劳。他们强烈
建议所有的Linux发行版都应该叫作GNU/Linux发行版。

在这个主题上引起了很多口水战,只输于vi与Emacs间的圣战了。本书的目的并
不是扇动这个已经炙手可热的讨论的战火,而是为新手阐明术语。当你看到
GNU/Linux时,它表示Linux发行版,但你看到Linux时,它可能指的是内核,也
可能指的是一个发行版。这还是很难区分的。特别的,因为GNU/Linux很饶舌,
所以一般不用。

\section{Slackware是什么}
\label{sec:intro:whatIsSlackware}

Slackware,由Patrick Volkerding于1992年末发起,并最早发布于1993年7年17
日,它是第一个得到广泛使用的Linux发行版。Volkerding最早认识Linux是在需
要为一个项目找一个廉价的LISP解释器的时候。那时候可用的发行版中有一个叫
SLS Linux,是Soft Landing Systems公司的产品。Volkerding用的是SLS Linux,
并在找到bugs时进行修复。最终,他决定将所有修复的bug合并到他自己私人的发行
版中,让他和他的朋友使用。这个私人发行版很快就小有名气,所以Volkerding
就决定将它命名为Slackware并对公众开放。在此过程中,Patrick为Slackware
添加了一些新的东西:一个很友好的基于菜单系统安装程序,以及软件包管理的
概念,包管理让用户在自己的系统上方便地添加、删除及更新软件包。

Slackware能成为现存的最古老的发行版,有诸多原因。例如它从不试图模仿
Windows,它尽可能地保持近似Unix。它并不试图用绚丽的指点GUI(图形用户接
口(Graphical User Interfaces))来隐藏一些操作。相反地,它让用户看到底层
的内容,以赐予用户最大的可控制性。它的开发并不是为了赶什么进度,新版本
只在准备好的时候才出。

Slackware适用于那些喜欢学习,喜欢通过配置自己的系统来实现想做的事的那些
人。Slackware的稳定性和简单性是多年来人们不断使用它的原因。Slackware目
前享有的美名是——一个坚固的服务器,一个严肃的工作站。你会发现,Slacware
可以运行几乎所有的窗口管理器或桌面环境,也可以不运行其中的任意一个。
Slackware提供了强大的服务,每个服务器能使用的地方,都能见到Slacware的
身影。Slacware的用户是Linux用户中满意度最高的。废话,我们当然会这么说
了。:\verb|^|)

\section{开源与自由软件}
\label{sec:intro:openSourceandFreeSoftware}

在Linux社区中,有两种主要的意识形态运动。自由软件运动(我们下面会说到)
的目标是使所有的软件都有免费的知识产权。这项运动的追随者们认为,知识产
权的限制阻碍了科技的发展并与社区的优点想违背。开源运动的目标与前者差不
多,但走了一条更实际的路线。这项运动的追随者们的论点是以使源代码自由可
得所带来的商业及技术的好处为基础,而不是一个精神与论理上的律条来推动自
由软件运动。

这个运动的底端是一些想更精密控制他们软件的组织。

自由软件运行是由自由软件基金会带头的,而该基金会正是为GNU项目筹款的。
自由软件远不止是一个意识形态。一个用烂的说法是``free\footnote{free在英
  语中即指自由,也指免费}指的是演讲的自由,而不是啤酒的免费。''。本质
上,自由软件是尝试同时保证用户与开发人员的一些权利。这些自由权利包括基
于任何目的地运行程序的权利,自由修改源码的权利,重新发布源码的权利以及
共享你所做的修改的权利。为了保证这些自由,他们创建了GNU通用公共许可
(GPL)。简单地说,GPL的内容是:任何人在发布一个遵守GPL许可的编译后的
程序时,也必须提供源代码,并且只要做出的修改也以源码的形式提供,就可以
对原先的源代码进行任意的修改。这就保证了一旦一个程序为社区``打开''了,
那么除非得到其中每部分代码(包括所做的修改)的作者的许可,那么这个程序
就不能被``关闭''。Linux下的程序绝大多数是遵守GPL许可的。

要注意一件事,GPL并没有说明有关价格的事。也许就像听起来很奇怪一样,你
可以对自由\footnote{free,有免费的意思}软件收费。许可中``自由''的部分
是针对源代码说的,而不是针对软件的价格据说的。(然而,一旦有人告诉你或
给你一个遵守GPL的编译后的软件,那么他也有义务给你软件的源码。)

另一个流行的许可是BSD许可,与GPL不同的是,BSD许可并不要求发布程序的源
码。遵守BSD许可的软件只要满足几个条件就可以重新以源码或二进制的形式发
布。程序作者的凭证并不作为程序某种形式上的广告。它也免除了作者任何因使
用该软件而造成损失带来的责任。Slackware中包含的许多软件是以BSD许可发布
的。

站在年轻的开源运动前线的,是称为Open Source Initiative的组织,它单独存
在,以获取开源软件的支持,也就是,软件总是可以在得到可执行程序的同时得
到它的源码。它们并不提供一个特定的许可,而是支持多种开源许可。

OSI背后的想法是通过让公司自己撰写自己的开源许可,并使许可通过OSI的认证
来争取更多的公司参与开源运动。许多公司同意发布源码,但并不想使用GPL许
可。由于不能改变GPL许可,所以OSI为它们提供了撰写自己的许可的机会,这个
许可最后由该组织认证。

虽然自由软件基金会与Open Source Initiative的工作是互相帮助的,但却不是
一个东西。自由软件基金会使用一个特殊的许可并提供该许可下的软件,而Open
Source Initiative则寻求所有的开源许可,包括自由软件基金会的那个。谈到
使人们自由获得源码的领域时通常分为两项运动,但这分立的两种不同意识形态
追求的目标是一致的,因此要信任双方所做出的努力。

%%% Local Variables: 
%%% mode: latex
%%% TeX-master: "../SlackGuide"
%%% End: 
