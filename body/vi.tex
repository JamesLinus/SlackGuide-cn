\chapter{VI}
\label{chap:vi}

\begin{flushleft}
\rule{\textwidth}{0.1pt}
\end{flushleft}

\texttt{vi}(1)是标准的Unix编辑器,尽管现在掌握它已经不如当年来得重要了,
但掌握\texttt{vi}曾经是,现在也是一个回报丰厚的目标。现在有许多
\texttt{vi}的版本(或称为克隆),包括\texttt{vi}、\texttt{elvis}、
\texttt{vile}及\texttt{vim}。任何的Unix版本中都至少包含它们中的一个,
当然,Linux亦是如此。所有这些\texttt{vi}克隆都有着一些共同的基本功能及
命令,所以只在学习其中的一种就能让我们同时掌握其它的。由于现在的Linux
版本以及Unix变种都包含了许多其它的文本编辑器,许多人已经不再使用
\texttt{vi}了,但它仍然是所有Unix及类Unix系统中最广泛存在的编辑器。掌
握\texttt{vi}意味着一旦坐在一台Unix机器前,我们至少不会对编辑器感到陌
生。

\texttt{vi}包含了许多强大的功能,包括语法高亮、代码重排、强大的搜索替
换机制、宏及其它一些功能。这些功能对于程序员、网页开发人员及类似工作的
人们吸引力尤大。而一些系统管理员甚至希望让\texttt{vi}与shell整合进行工
作。

在Slackware Linux中,默认的\texttt{vi}版本是\texttt{elvis},但同时还包
括了\texttt{vim}和\texttt{gvim},当然,前提是你安装了相关的包。而
\texttt{gvim}则是在X环境下的\texttt{vim}版本,它包含了工具栏、可分享的
菜单及对话框等等。

学习\texttt{vi}最好的方法当然就是练习,但教程也是极为重要的,幸运的是,
\texttt{vi}自身就包含了良好的文档,可以在\texttt{vi}中通过
\texttt{:help}打开。而这些文档也在良好的中文翻译。请查阅相关的内容。笔
者还向你推荐O'reilly出版社的《学习VI和VIM编辑器》一书。本章中只对
\texttt{vi}的基本功能进行描述。

\section{启动vi}
\label{sec:vi:start}
\texttt{vi}可以通过许多方式从命令行中启动。最简单的方式就是:
\begin{Verbatim}[frame=single,commandchars=\\\{\}]
\% \textbf{vi}
\end{Verbatim}
\begin{figure}[ht]
  \centering
  \begin{Verbatim}[frame=single,commandchars=\\\{\}]
~
~
~
~
~
~
~
~
~
~
~
                                                             Command
\end{Verbatim}

  \caption{VI启动界面}
  \label{fig:vi-start}
\end{figure}
这个命令会启动\texttt{vi},并开启一个空的缓冲区,如图
\ref{fig:vi-start}所示。现在你可能就开始输入了,并且期望我们的输入能立
即反应到屏幕上。但这时候发现,事实并非如此,屏幕可能连反应都没有,也可
能出现乱七八糟的字符。原因很简单,\texttt{vi}是``有模式''的编辑器,而
你现在正处于它的``命令模式''下。\texttt{vi}有两个模式:``命令模式''和
``插入模式''。命令模式是默认的模式,在命令模式中,我们在键盘上敲的所有
键都会被\texttt{vi}解释为一个命令,这些命令的含义可以是移动光标、删除
字符、复制粘贴,搜索等等。而在插入模式,所有我们敲的字符都会直接反应在
屏幕上,这个模式与我们传统的习惯相同。关于模式的详细讨论请见第
\ref{sec:vi:modes}节。

要退出\texttt{vi},只要在\texttt{vi}中输入下列键:
\begin{Verbatim}[frame=single,commandchars=\\\{\}]
\textbf{:q}
\end{Verbatim}
假设我们没有对文件进行任何改动,则\texttt{vi}正常退出。如果做了一些改
动,则\texttt{vi}会发出警告,告诉我们有未保存的修改,并告诉我们如果想
放弃修改应该怎么做。而放弃修改一般意味着我们要对退出命令作些解释,在
``q''之后加上``!''即表示丢弃所有未保存的修改而直接退出:
\begin{Verbatim}[frame=single,commandchars=\\\{\}]
\textbf{:q!}
\end{Verbatim}
`!'一般意味着强制执行一些动作。我们会在之后解释键组合时进一步说明。

我们也可以在启动\texttt{vi}时同时指定打开某个文件。例如,我们可以在启
动\texttt{vi}时同时打开文件\path{/etc/resolv.conf}:
\begin{Verbatim}[frame=single,commandchars=\\\{\}]
\% \textbf{vi /etc/resolv.conf}
\end{Verbatim}

最后,\texttt{vi}可以在启动时就跳到一个文件的某个特定行。这对于程序员
在调试时知道某行出错而进行修改时很有用。例如,我们可以启动\texttt{vi},
并打开\path{/usr/src/linux/init/main.c},并跳到第47行,如下:
\begin{Verbatim}[frame=single,commandchars=\\\{\}]
\% \textbf{vi +47 /usr/src/linux/init/main.c}
\end{Verbatim}
\texttt{vi}会打开该文件并将光标放在指定的行上。而如果指定的行大于文件
的行数,则\texttt{vi}会将光标放在文件的最后一行。这对于程序员来说极为
有用,这意味着他们可以不用搜索就跳转到指定的行。


\section{模式}
\label{sec:vi:modes}
模式就像金箍棒,用不惯的人觉得是废物,用惯的人就知道是神器。
\texttt{vi}有许多模式,用于完成各种任务。从现在开始,我们就可以学习如
何处理文本、在文件中移动、保存、退出以及变换切换模式。编辑文本则是在插
入模式下完成的。我们可以使用很多个键来在模式中切换,下面就要讲到:

\subsection{命令模式}
\label{sec:vi:modes:command}
我们默认进入的就是命令模式。在本模式下,我们不能直接键入文本或是对已有
文本直接修改。然而,在此模式下,我们可以处理文本、搜索、退出、保存、载
入新文件及其它的一些工作。本节中只对命令模式本身进行介绍,具体的命令请
参见第\ref{sec:vi:keys}节。

也许在命令模式下最常用的命令就是用来切换到插入模式。这可以通过按下
\texttt{i}键完成。进入插入模式后,光标的形状发生了改变,在屏幕下方会出
现\texttt{--INSERT--}字样(注意,\texttt{vi}的各版本行为不尽相同,如
\texttt{elvis}会在下方显示\texttt{Input}字样)。现在,我们按下的所有键,
都会直接输入当前的缓冲区中,显示在屏幕上。如果想回到命令模式,请按下
\texttt{Esc}键。

要想在文件中进行移动,也是在命令模式下完成的。一些系统中,我们是使用方
向键进行移动的,\texttt{vi}中也保留了这个设定,但在另一些系统中,我们
需要使用更古老的``hjkl''键进行移动。下面是这些命令的简单介绍:
\begin{table}[ht]
  \centering
  \begin{tabular}{c|l}
  \hline\hline
  h & 向左移动一个字符 \\
  j & 向下移动一个字符 \\
  k & 向上移动一个字符 \\
  l & 向右移动一个字符 \\ 
  \hline\hline
\end{tabular}
\end{table}
在命令模式下按下相应的键就可以在文本中进行移动。我们之后会看到,这些键
能加上数字作为参数,因此移动起来更有效率。

我们在命令模式下看到的许多命令是带冒号(`:')的。例如之前用过的
\texttt{:q}。冒号意味着这是一个命令,而后面则是命令的本尊,如
\texttt{q}就是告诉\texttt{vi}退出的命令。另外一些命令是一些可选的数字,
之后跟上一个字符。这些命令之前没有冒号,且一般用于处理文本。

例如,在文件中删除一行是通过\texttt{dd}完成的。它会删除当前光标所在的
行,而使用\texttt{4dd}这个命令则告诉\texttt{vi}从当前光标开始,向下删
除4行。一般来说,数字是告诉\texttt{vi}执行相同的命令多少次数。

我们也可以在移动的按键前加上数字,如\texttt{10k}表示向下移动10行。

命令模式也可以用来复制粘贴、输入文本及将其它文件的内容读入当前缓冲区。
复制是通过\texttt{y}键来实现的(y表示yank)。而复制当前的行则是按下
\texttt{yy}键完成的,当然,同前面一样,可以在这些键前加上数字来指定一
次复制多少行。复制之后,在要粘贴的位置按下\texttt{p}键,则复制的内容复
制到当前光标之后。

剪切的功能是通过\texttt{dd}完成的,剪切后使用\texttt{p}键就能将剪切的
内容粘贴到指定位置。而读取另一个文件也很简单,只要输入\texttt{:r
  filename}就可以在当前行的下一行插入\texttt{filename}文件的内容。更高
端的\texttt{vi}克隆还有补全功能,像shell一样。

最后要介绍的是查找。命令模式即支持简单的查找,也支持异常复杂的查找替换
功能,这些功能要使用加强版的正则表达式。由于完整介绍正则表达式已经超出
本书的范围,所以这里只介绍最简单的查找。

要执行简单的查找,先按下\texttt{/}键,之后输入要查找的内容。
\texttt{vi}会从当前光标搜索至文件末尾,直到找到一个匹配为止。注意,不
完全匹配也会让\texttt{vi}停下。例如,我们要查找``the'',但\texttt{vi}
在找到``then''、``therefore''时也会停下。这是因为这两个词也匹配了``the''。

在\texttt{vi}找到第一个匹配后,我们可以输入\texttt{/}并按回车来查找下
一个匹配,但更方便的是按\texttt{n}键,代表下一个匹配。我们还可以使用
\texttt{?}代替\texttt{/}来执行反向搜索,即从当前光标向文件头搜索。

\subsection{插入模式}
\label{sec:vi:modes:insert}
插入、替换文本是在插入模式完成的。正如之前讨论的,可以在命令模式下输入
\texttt{i}键切换到插入模式。之后,所有输入的内容都会直接输入到当前缓冲
区中。而按下\texttt{Esc}键则会回到命令模式。

替换文本有好几种方法。在命令模式下,我们可以输入\texttt{r},它允许我们
替换当前光标下的一个字符,之后又直接回到命令模式。而键入\texttt{R}则允
许我们替换任意字符。要退出``替换模式'',键入\texttt{Esc}即可。

还有一种可以在输入和替换间切换,按下键盘上的\texttt{Ins}键,会直接从命
令模式进入插入模式,而进入插入模式后,它的作用就成了在插入与替换间的切
换。按一下就进入替换模式,再按一下又回到插入模式。

\section{打开文件}
\label{sec:vi:open}
\texttt{vi}允许我们从命令模式下打开一个文件,也允许我们直接在命令行启
动时指定,这在前面介绍过了。命令模式下,要打开\path{/etc/lilo.conf}文
件:
\begin{Verbatim}[frame=single,commandchars=\\\{\}]
\textbf{:e /etc/lilo.conf}
\end{Verbatim}
如果我们对当前的缓冲做了修改但还没有保存,则\texttt{vi}会提示尚未保存,
并阻止新文件的打开。当然,我们还是可以通过输入\texttt{:e!}命令来强制打
开新文件,并且不保存当前缓冲的修改。一般,可以使用叹号(`!')来忽略
\texttt{vi}的警告,强制执行某些命令。

如果你想重新载入当前文件,那么可以直接输入\texttt{e!},如果我们把文件
改乱了,那么这是非常有用的命令。

而一些\texttt{vi}克隆(如\texttt{vim})允许我们同时打开多个缓冲。例如,
我们在当前打开文件的情况下,想打开主目录下的\texttt{09-vi.sgml}文件,那
么我们可以输入:
\begin{Verbatim}[frame=single,commandchars=\\\{\}]
\textbf{:split ~/09-vi.sgml}
\end{Verbatim}

这时屏幕会分成上下两部分,而新打开的文件会显示在上半部分,原先的文件则
在下半部分。而这些命令可以认为是模仿\texttt{Emacs}的开始。当然,了解这
些命令的最好方法是查看这些\texttt{vi}克隆的帮助文件。当然,还有许多克
隆不支持分屏,所以可能连用都不能用。

\section{保存文件}
\label{sec:vi:save}
在\texttt{vi}中保存文件有许多方法。如,我们想将当前的缓冲区保存为
\texttt{randomness},我们可以在命令模式下输入:
\begin{Verbatim}[frame=single,commandchars=\\\{\}]
\textbf{:w randomness}
\end{Verbatim}

一旦我们保存过一次,下次保存时就不需要再输入文件名了,只要输入
\texttt{:w}就可以了,期间对文件进行的修改都会保存到文件中。如果我们想
保存文件并退出(很经常的操作),我们可以输入\texttt{:wq},这告诉
\texttt{vi},先保存当前文件,之后退出到shell中。

有时,我们想将修改保存到只读文件中,我们就必须在保存命令后加上叹号,如:
\begin{Verbatim}[frame=single,commandchars=\\\{\}]
\textbf{:w!}
\end{Verbatim}

当然,有时这个命令还是无效(如:我们想修改一个别人拥有的文件)。如果我
们强烈地想修改那个文件,那么可以以\texttt{root}用户或文件的属主(推荐)
登陆对文件进行修改。

\section{退出vi}
\label{sec:vi:quit}
退出\texttt{vi}的一个方法是\texttt{:wq},我们就可以在退出前保存当前缓
冲。我们也可以不保存就退出\texttt{:q}或(更常见的)\texttt{:q!}。后面
这个命令用于我们做了修改但不想保存的情况。

有一些情况下,我们的机器崩溃或\texttt{vi}崩溃了。这时\texttt{elvis}和
\texttt{vim}都做了一些工作来将损失减到最小。这两个编辑器都会打开的缓冲
保存为一个临时文件,该文件的文件名与原文件相似,只是以点号(`.')开头,
所以它是个隐藏文件。

这个临时文件在编辑器正常退出后会删除,这意味着如果编辑器崩溃时,这些文
件仍存在,那么如果我们在崩溃后再次编辑原来的文件,编辑器会提示我们应该
怎么做。多数情况下,我们可以恢复很大一部分的工作。\texttt{elvis}还会给
你发封邮件告诉你它有一份当前备份。

\section{vi配置}
\label{sec:vi:configuration}
\texttt{vi}可以通过多种方式进行配置。

我们可以根据自己的喜好在启动\texttt{vi}后输入一堆配置命令。根据我们使
用的\texttt{vi}克隆,我们可以作些设置来方便编程(如语法高亮、自动缩进
等等),还可以编写宏来自动完成一些任务,启用文本替换等等。

几乎所有这些设置命令都可以放入主目录下的一个配置文件中。\texttt{elvis}
使用的配置文件为\texttt{.exrc},而\texttt{vim}则使用\texttt{.vimrc}文
件。绝大多数配置命令都可以放在配置文件中,包括安装信息、文本替换、宏等
等。

由于不同的编辑器选项不同,在这里要讨论所有的选项不太现实,这方面的内容
请参考你喜欢的编辑器的man手册、官网、帮助文件等等。一些编辑器(如
\texttt{vim}及\texttt{elvis})自带了帮助系统,通过输入命令
\texttt{:help}就可以调用。当然《学习VI和VIM编辑器》一书也是很好的参考。

Linux中的许多软件内部会默认使用\texttt{vi}来编辑文件。例如,编辑
crontabs会默认使用\texttt{vi},如果你不喜欢\texttt{vi},而想用其它的编
辑器,可以将\texttt{VISUAL}环境变量设置成你喜欢的编辑器。如果你希望每
次登陆后使用的默认编辑器都一致,那么将对\texttt{VISUAL}的设置加入
\path{.bash_profile}或\path{.bashrc}文件中。

\section{vi键绑定}
\label{sec:vi:keys}
本节是许多常用\texttt{vi}命令的快速参考,有些命令之前介绍了,有些则是
新的。
\begin{table}[ht]
  \centering
  \begin{tabular}{c|l}
  \hline\hline
  键 & 操作 \\ \hline
  h,j,k,l & 向左、下、上、右移动一个字符 \\
  \$ & 移动到行末 \\
  $\wedge$ & 移动到行首 \\
  G & 移动到文件末尾 \\
  :1 & 移动到文件开头 \\
  :47 & 移动到第47行 \\
  \hline\hline
\end{tabular}

  \caption{VI——移动命令}
  \label{tab:vi-movement}
\end{table}

\begin{table}[htpb]
  \centering
  \begin{tabular}{c|l}
    \hline \hline
    键 & 操作 \\ \hline
    dd & 删除一行 \\
    5dd & 删除5行 \\
    r & 替换当前光标下的一个字符 \\
    x & 删除一个字符 \\
    10x & 删除10个字符 \\
    u & 撤消最后一次更改 \\
    J & 连接当前行和下一行 \\
    \%s/old/new/g & 将全文中所有的old替换为new \\
    \hline \hline
  \end{tabular}
  \caption{VI——编辑命令}
  \label{tab:vi-edit}
\end{table}

\begin{table}[htpb]
  \centering
  \begin{tabular}{c|l}
    \hline \hline
    键 & 操作 \\ \hline
    /asdf & 搜索asdf \\
    ?asdf & 反向搜索asdf \\
    / & 向前重复最后一次搜索 \\
    ? & 反向重复最后一次搜索 \\
    n & 重复最后一次搜索,方向也相同 \\
    N & 重复最后一次搜索,方向相反 \\
    \hline\hline
  \end{tabular}
  \caption{VI——搜索命令}
  \label{tab:vi-search}
\end{table}

\begin{table}[htpb]
  \centering
  \begin{tabular}{c|l}
    \hline \hline
    键 & 操作 \\ \hline
    :q & 退出 \\
    :q! & 不保存就退出 \\
    :wq & 保存并退出 \\
    :w & 保存当前修改 \\
    :e! & 放弃未保存的所有修改 \\
    :w hejaz & 将当前缓冲保存为jejaz文件 \\
    :r asdf & 读取asdf文件的内容插入当前缓冲 \\
    :r !ls & 读取ls命令的输入插入当前缓冲 \\
    \hline \hline
  \end{tabular}
  \caption{VI——保存及退出}
  \label{tab:vi-saveAndQuit}
\end{table}
%%% Local Variables: 
%%% mode: latex
%%% TeX-master: "../SlackGuide"
%%% End: 
